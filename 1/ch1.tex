\section{Leerlaufversuch}
Der Leerlaufversuch dient der Bestimmung der Leelraufverlusten,  die aus Eisenverlusten (Hystere- und Wirbelstromverlusten), sowie Zusatzverlusten  (Streuflüsse  bei  unsym.   Bauformen)  bestehen. DIe Kupferverluste können bei dieser Messung vernachlässigt werden.\\
Für die Durchführung ist nur die Primärseite des Transformators veraschaltet, während die Sekundärseite noch nicht verschalten ist bzw. offen ist. 
Bei der Messung werden alle Außenleiterspannungen und Strangströme der Primärseite gemessen und für die Berechnungen die jeweils gemittelten Werte genutzt.
\input{\currfiledir leerlauf}
\input{\currfiledir leerlauf_spannung}
\input{\currfiledir zeitverlauf}

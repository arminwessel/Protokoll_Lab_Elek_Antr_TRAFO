\begin{figure}
	\begin{minipage}{.45\textwidth}
		\centering
		\begin{tikzpicture}[>=triangle 45,thick,node distance=0.5cm]
		
		\path (0,0) coordinate (origin1);
		\path (90:3cm) coordinate (1U);
		\path (2*120+90:3cm) coordinate (1V);
		\path (1*120+90:3cm) coordinate (1W);
		\path (0,-6.5cm) coordinate (origin2);
		\path (origin2) ++(-120:3cm) coordinate (2U);
		\path (origin2) ++(-120+2*120:3cm) coordinate (2V);
		\path (origin2) ++(-120+1*120:3cm) coordinate (2W);
		\path (1.7cm,1.7cm) coordinate (U1U1V);
		\path (-1.7cm,1.7cm) coordinate (U1W1U);
		\path (0cm,-1.5cm) coordinate (U1V1W);
		\path (origin2) ++(-150:1.73205cm) coordinate (UAdrittel);
		\path (origin2) ++(1.0cm,0.5cm) coordinate (Ud);
		\path (origin2) ++(1.2cm,2.2cm) coordinate (Ue);
		
		\draw [->] (origin1) -- node[right] {$U_{1U1N}=U_a$} (1U);
		\draw [->] (origin1) -- (1V);
		\draw [->] (origin1) -- node[above left] {$U_{1W1N}=U_c$} (1W);
		
		\draw [->] (origin2) -- (2U);
		\draw [->] (origin2) -- (2V);
		\draw [->] (origin2) -- (2W);
		
		\draw [->] (origin2) -- node[above left] {$U_f=U_i$} (UAdrittel);
		\draw [->] (UAdrittel) -- node[left] {$-U_d=-U_g$} (2U);
		
		\node [above of =1U] {1U};
		\node [below right of =1V] {1V};
		\node [below left of =1W] {1W};
		\node [below left of =2U] {2U};
		\node [above left of =2V] {2V};
		\node [below right of =2W] {2W};
		
		\end{tikzpicture}
		\caption{Zeigerdiagramm}
	\end{minipage}
	~
	\begin{minipage}{.45\textwidth}
		\centering
		\begin{circuitikz}
			
			%\draw [help lines] (-1,3) grid (5,-13); %Zeichnet Raster und vereinfacht damit das Zeichnen
			
			%U
			\draw (0,1) node[above=5mm] {$1U$} to[short,o-] (0,0)
			to[american inductor] (0,-3);
			\draw (0,-0.8) node[right=1.5mm] {$\bullet$};
			
			\draw (0,-5) to[american inductor] (0,-7.5);
			\draw (0,-5.7) node[right=1.5mm] {$\bullet$};
			
			\draw (0,-8.5) to[american inductor,-o] node[below=5mm] {$2U$} (0,-11);
			\draw (0,-9.2) node[right=1.5mm] {$\bullet$};
			
			%V
			\draw (2,1) node[above=5mm] {$1V$} to[short,o-] (2,0)
			to[american inductor] (2,-3);
			\draw (2,-0.8) node[right=1.5mm] {$\bullet$};
			
			\draw (2,-5) to[american inductor] (2,-7.5);
			\draw (2,-5.7) node[right=1.5mm] {$\bullet$};
			
			\draw (2,-8.5) to[american inductor,-o] node[below=5mm] {$2V$} (2,-11);
			\draw (2,-9.2) node[right=1.5mm] {$\bullet$};
			
			%W
			\draw (4,1) node[above=5mm] {$1W$} to[short,o-] (4,0)
			to[american inductor] (4,-3);
			\draw (4,-0.8) node[right=1.5mm] {$\bullet$};
			
			\draw (4,-5) to[american inductor] (4,-7.5);
			\draw (4,-5.7) node[right=1.5mm] {$\bullet$};
			
			\draw (4,-8.5) to[american inductor,-o] node[below=5mm] {$2W$} (4,-11);
			\draw (4,-9.2) node[right=1.5mm] {$\bullet$};
			
			%N
			\draw (-1.5,-10) to[short,-o] node[below=5mm] {$2N$} (-1.5,-11);
			
			%Verbindung
			%Primär
			\draw (0,-3) -- (4,-3);
			
			%Sekundär
			\draw (0,-5) -- (1,-5)
			|- (2,-8.5);
			\draw (2,-5) -- (3,-5)
			|- (4,-8.5);
			\draw (4,-5) |- (-1,-4)
			|- (0,-8.5);
			%Neutralleiter
			\draw (-1.5,-10) |- (0,-8)
			to[short,*-] (2,-8)
			to[short,*-] (4,-8);
			\draw (0,-8) -- (0,-7);
			\draw (2,-8) -- (2,-7);
			\draw (4,-8) -- (4,-7);
			
			%Spannungspfeile
			\draw (-0.2,0) to [open, v>=$U_a$] (-0.2,-3);
			\draw (1.8,0) to [open, v>=$U_b$] (1.8,-3);
			\draw (3.8,0) to [open, v>=$U_c$] (3.8,-3);
			
			\draw (-0.2,-5) to [open, v>=$U_d$] (-0.2,-7.5);
			\draw (1.8,-5) to [open, v>=$U_e$] (1.8,-7.5);
			\draw (3.8,-5) to [open, v>=$U_f$] (3.8,-7.5);
			
			\draw (-0.2,-8.5) to [open, v>=$U_g$] (-0.2,-11);
			\draw (1.8,-8.5) to [open, v>=$U_h$] (1.8,-11);
			\draw (3.8,-8.5) to [open, v>=$U_i$] (3.8,-11);
			
		\end{circuitikz}
		\caption{Schaltung}
	\end{minipage}%
\end{figure}

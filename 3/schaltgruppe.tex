\begin{figure}
\centering
\begin{tikzpicture}[scale=1]

% W als Ursprung gewählt
% U und w sind kurzgeschlossen

% V und W beliebig festlegen (als Dreiecksbasis)
\coordinate (W1) at (0, 0);
\coordinate (V1) at (10, 0);

% Konstruktion primär
\draw[name path=Warc, dotted] (70:10) arc [start angle=70, end angle=50, radius=10];
\draw[name path=Varc, dotted] ([shift=(V1)] 130:10) arc [start angle=130, end angle=110, radius=10];
\path[name intersections={of=Warc and Varc,name=U1}];
\coordinate (U1) at (U1-1);
\coordinate (U2) at (U1-1);
\coordinate (N) at ($1/3*(U1)+1/3*(V1)+1/3*(W1)$);

% Konstruktion sekundär
\draw[name path=warc, dotted] ([shift=(U2)] -20:6.66) arc [start angle=-20, end angle=-40, radius=6.66];
\draw[name path=uwarc, dotted] ([shift=(U2)] -100:6.68) arc [start angle=-100, end angle=-80, radius=6.68];
\draw[name path=Vuarc, dotted] ([shift=(V1)] 75:5.4) arc [start angle=75, end angle=90, radius=5.4];
\draw[name path=Vvarc, dotted] ([shift=(V1)] 155:5.4) arc [start angle=155, end angle=170, radius=5.4];
\path[name intersections={of=warc and Vuarc,name=V2}];
\path[name intersections={of=uwarc and Vvarc,name=W2}];
\coordinate (V2) at (V2-1);
\coordinate (W2) at (W2-1);
\coordinate (n) at ($1/3*(V2)+1/3*(W2)+1/3*(U2)$);

% Punkte
\draw (N) node[anchor=south east, outer sep=3pt] {N};
\draw (U1) node[anchor=south, outer sep=5pt] {U1, U2};
\draw (V1) node[anchor=north west] {V1};
\draw (W1) node[anchor=north east] {W1};
\draw (V2) node[anchor=west] {V2};
\draw (W2) node[anchor=north] {W2};

% Verbindungen und Spannungspfeile
\draw[gray] (V1) -- (W1);
\draw[gray] (W1) -- (U1);
\draw[gray] (U1) -- (V1);
\draw[->,name path=UN] (U1) -- (N);
\draw[->] (V1) -- (N);
\draw[->] (W1) -- (N);
\draw[->] (V2) -- (W2);
\draw[->] (W2) -- (U2);
\draw[->] (U2) -- (V2);

% Kreisbogenbeschriftungen
\draw[<-] ([shift=(U2)] -35:6.66) -- ([shift=(U2)] -35:7.16) node[anchor=west] {$U_\mathrm{U1V2}$};
\draw[<-] ([shift=(U2)] -80:6.68) -- ([shift=(U2)] -80:7.18) node[anchor=north] {$U_\mathrm{U1W2}$};
\draw[<-] ([shift=(V1)] 88:5.4) -- ([shift=(V1)] 88:5.9) node[anchor=south] {$U_\mathrm{V1V2}$};
\draw[<-] ([shift=(V1)] 168:5.4) -- ([shift=(V1)] 168:5.9) node[anchor=south] {$U_\mathrm{V1W2}$};
\draw[<-] ([shift=(V1)] 128:10) -- ([shift=(V1)] 128:10.5) node[anchor=south] {$U_\mathrm{U1V1}$};
\draw[<-] ([shift=(W1)] 68:10) -- ([shift=(W1)] 68:10.5) node[anchor=south] {$U_\mathrm{W1U1}$};

% Stundenziffer (ausgeblendet)
%\begin{scope}
%\draw[name path=un] (u) -- ($(n)!-0.6!(u)$);
%\path[name intersections={of=UN and un,name=helper}];
%\coordinate (helper) at (helper-1);
%\path[clip] (U) -- (helper) -- (u) -- cycle;
%\draw (helper) circle (0.8);
%\draw (helper) node[anchor=south west] {\SI{90}{\degree}};
%\end{scope}

\end{tikzpicture}
\caption{Bestimmung der Schaltgruppe durch Messungen}
\end{figure}

\begin{figure}[h!]
\centering
\begin{tikzpicture}[scale=1]

% W als Ursprung gewählt
% U und w sind kurzgeschlossen

% V und W beliebig festlegen (als Dreiecksbasis)
\coordinate (1W) at (0, 0);
\coordinate (1V) at (10, 0);

% Konstruktion primär
\draw[name path=Warc, dotted] (70:10) arc [start angle=70, end angle=50, radius=10];
\draw[name path=Varc, dotted] ([shift=(1V)] 130:10) arc [start angle=130, end angle=110, radius=10];
\path[name intersections={of=Warc and Varc,name=1U}];
\coordinate (1U) at (1U-1);
\coordinate (2U) at (1U-1);
\coordinate (N) at ($1/3*(1U)+1/3*(1V)+1/3*(1W)$);

% Konstruktion sekundär
\draw[name path=warc, dotted] ([shift=(2U)] -20:6.66) arc [start angle=-20, end angle=-40, radius=6.66];
\draw[name path=uwarc, dotted] ([shift=(2U)] -100:6.68) arc [start angle=-100, end angle=-80, radius=6.68];
\draw[name path=Vuarc, dotted] ([shift=(1V)] 75:5.4) arc [start angle=75, end angle=90, radius=5.4];
\draw[name path=Vvarc, dotted] ([shift=(1V)] 155:5.4) arc [start angle=155, end angle=170, radius=5.4];
\path[name intersections={of=warc and Vuarc,name=2V}];
\path[name intersections={of=uwarc and Vvarc,name=2W}];
\coordinate (2V) at (2V-1);
\coordinate (2W) at (2W-1);
\coordinate (n) at ($1/3*(2V)+1/3*(2W)+1/3*(2U)$);

% Punkte
\draw (N) node[anchor=south east, outer sep=3pt] {N};
\draw (1U) node[anchor=south, outer sep=5pt] {1U, 2U};
\draw (1V) node[anchor=north west] {1V};
\draw (1W) node[anchor=north east] {1W};
\draw (2V) node[anchor=west] {2V};
\draw (2W) node[anchor=north] {2W};

% Verbindungen und Spannungspfeile
\draw[gray] (1V) -- (1W);
\draw[gray] (1W) -- (1U);
\draw[gray] (1U) -- (1V);
\draw[->,name path=UN] (1U) -- (N);
\draw[->] (1V) -- (N);
\draw[->] (1W) -- (N);
\draw[->] (2V) -- (2W);
\draw[->] (2W) -- (2U);
\draw[->] (2U) -- (2V);

% Kreisbogenbeschriftungen
\draw[<-] ([shift=(2U)] -35:6.66) -- ([shift=(2U)] -35:7.16) node[anchor=west] {$U_\mathrm{1U2V}$};
\draw[<-] ([shift=(2U)] -80:6.68) -- ([shift=(2U)] -80:7.18) node[anchor=north] {$U_\mathrm{1U2W}$};
\draw[<-] ([shift=(1V)] 88:5.4) -- ([shift=(1V)] 88:5.9) node[anchor=south] {$U_\mathrm{1V2V}$};
\draw[<-] ([shift=(1V)] 168:5.4) -- ([shift=(1V)] 168:5.9) node[anchor=south] {$U_\mathrm{1V2W}$};
\draw[<-] ([shift=(1V)] 128:10) -- ([shift=(1V)] 128:10.5) node[anchor=south] {$U_\mathrm{1U1V}$};
\draw[<-] ([shift=(1W)] 68:10) -- ([shift=(1W)] 68:10.5) node[anchor=south] {$U_\mathrm{1W1U}$};

% Stundenziffer (ausgeblendet)
%\begin{scope}
%\draw[name path=un] (u) -- ($(n)!-0.6!(u)$);
%\path[name intersections={of=UN and un,name=helper}];
%\coordinate (helper) at (helper-1);
%\path[clip] (U) -- (helper) -- (u) -- cycle;
%\draw (helper) circle (0.8);
%\draw (helper) node[anchor=south west] {\SI{90}{\degree}};
%\end{scope}

\end{tikzpicture}
\caption{Bestimmung der Schaltgruppe durch Messungen}
\label{fig:schaltgruppe_messung}
\end{figure}

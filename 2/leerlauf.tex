\begin{figure}
    \centering
    \begin{tikzpicture}
    \begin{axis}[
       width=\textwidth,
       height= 12cm,
	   xlabel=Leerlaufstrom $I_0$,
	   x unit = \si{\ampere},
	   ylabel= {Scheinleistung $S_0$, Wirkleistung $P_0$},
	   y unit = {\si{\volt\ampere}, \si{\watt}},
	   grid=major,
        xmin=0,
        xmax=30,
		ymin=0,
		ymax=25000]
		\addplot[smooth,blue,line width=0.5mm] table[x=I_0, y=P_0, header=has colnames,col sep=comma] {\currfiledir leerlauf.csv};
		\label{tikz_leerlauf1}
		\addplot[smooth,red,line width=0.5mm] table[x=I_0, y=S_0, header=has colnames,col sep=comma] {\currfiledir leerlauf.csv};
		\label{tikz_leerlauf2}
		\end{axis}
		\begin{axis}[
		width=\textwidth,
		height=12cm,
		axis x line = none,
	%	ylabel pos= right,
		yticklabel pos = right,
		ylabel near ticks,
		ylabel= {Leistungsfaktor $\cos(\varphi)$},
		y unit = 1,
		xmin = 0,
		xmax = 30,
		ymin = 0,
		ymax = 1,
		legend style={
            at={(0.1,0.3)},
            anchor=west}
            ]
            \addlegendimage{/pgfplots/refstyle=tikz_leerlauf1}\addlegendentry{Wirkleistung $P_0$}
            \addlegendimage{/pgfplots/refstyle=tikz_leerlauf2}\addlegendentry{Scheinleistung $S_0$}
            \addplot[smooth,blue,dashed, line width=0.5mm] table[x=I_0, y=cosphi, header=has colnames,col sep=comma] {\currfiledir leerlauf.csv};
            \addlegendentry{Leistungsfaktor $\cos(\varphi)$}
            
        \end{axis}
	\end{tikzpicture}
	\caption{Leerlaufversuch; durchgezogene Linien: linke Skala, gestrichelte Linien: rechte Skala}
\end{figure}

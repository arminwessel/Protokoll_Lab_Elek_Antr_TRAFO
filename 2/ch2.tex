\section{Einschaltvorgang}
Beim Einschaltvorgang wird der Stromverlauf aller Stränge mit einem Oszilloskop gemessen. Nachdem der Transformator an die 3 phasig Versorgungspannung (Nennspannung) gelegt wird, kann der Einschwingvorgang beobachtet werden. Der dabei gemessene Strom hängt von der zum Einschaltzeitpunkt anliegenden Strangspannung ab.\\
Bei einem Spannungsnulldurchgang beim Einschalten, stellt sich ein magnetischer Fluss ein, der proportional der Spannungszeitfläche ist. Dies entspricht genau dem zweifachen stationärem Fluss. Die so entstehenden Flussspitzen führen wegen der Sätttigung des Eisens zu stark verzerrten Strömen mit Spitzenwerten von bis zu 15-fachen Nennstrom (siehe Abbildung \ref{}). Dieser Zussammenhang ist über die Magnetisierungskennlinie gegeben. \\
Der ideele EInschaltzeitpunkt (siehe Abbildung \ref{}) liegt bei einem Maxima der Versorgungsspannung und resultiert in keiner extremen Stromspitze.
\input{\currfiledir bestcase}
\input{\currfiledir worstcase}